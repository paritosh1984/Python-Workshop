\documentclass{beamer}
\usepackage[utf8]{inputenc}
\usepackage{graphicx}
\usepackage{subcaption}
\usepackage{listings}
%\usepackage{subfig}

\usetheme{Madrid}
\usecolortheme{seahorse}
\usefonttheme{serif} 
 
 
%Information to be included in the title page:
\title{Tree Traversals}
\author{Vivek Singh}
\institute{Information Systems Decision Sciences (ISDS)\\
MUMA College of Business\\
University of South Florida \\
Tampa, Florida}
\date{2018}
 
 
\begin{document}
\frame{\titlepage}
\begin{frame}
\frametitle{Introduction}
With a solid understanding of trees and advantages of tree data structures, we can look at tree traversals.
\begin{itemize}
\item There are three commonly used patterns to visit all the nodes in a tree.
\item They differ by the order in which the nodes are visited.
\item The three traversals we will look at
are called preorder, inorder, and postorder.
\item Preorder - In a preorder traversal, we visit the root node first, then recursively do a preorder traversal of the left subtree, followed by a recursive preorder traversal of the right subtree.
\item Inorder - In an inorder traversal, we recursively do an inorder traversal on the left subtree, visit the root node, and finally do a recursive inorder traversal of the right subtree.
\item Postorder - In a postorder traversal, we recursively do a postorder traversal of the left subtree and the right subtree followed by a visit to the root node.
\end{itemize}
\end{frame}

\begin{frame}[fragile]
\frametitle{Preorder Traversal}
We could have an external method as follows:
\begin{lstlisting}[language=Python]
def preorder(tree):
    if tree:
        print(tree.get_root_val())
        preorder(tree.get_left_child())
        preorder(tree.get_right_child())
\end{lstlisting}

An internal method could be implemented as follows:
\begin{lstlisting}[language=Python]
def preorder(self):
    print(self.key)
    if self.left_child:
        self.left.preorder()
    if self.right_child:
        self.right.preorder()
\end{lstlisting}
\end{frame}

\begin{frame}[fragile]
\frametitle{Preorder and Inorder Traversal}
The algorithm for the postorder traversal is nearly identical to preorder except that we move the call to print to the end of the function.
\begin{lstlisting}[language=Python]
def preorder(tree):
    if tree:
        print(tree.get_root_val())
        preorder(tree.get_left_child())
        preorder(tree.get_right_child())
\end{lstlisting}

In the inorder traversal we visit the left subtree, followed by the root, and finally the right subtree.
\begin{lstlisting}[language=Python]
def inorder(tree):
    if tree != None:
        inorder(tree.get_left_child())
        print(tree.get_root_val())
        inorder(tree.get_right_child())
\end{lstlisting}
\end{frame}

\begin{frame}
\frametitle{Summary}
\begin{itemize}
\item We learned Tree traversal and why it is important.
\item We also learned  the three commonly used patterns to traverse tree data structures.
\begin{itemize}
\item Preorder,
\item Inorder and
\item Postorder.
\end{itemize}
\end{itemize}
\end{frame}
\end{document}
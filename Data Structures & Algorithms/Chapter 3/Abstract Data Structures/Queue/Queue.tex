\documentclass{beamer}
\usepackage[utf8]{inputenc}
\usepackage{graphicx}
\usepackage{subcaption}
\usepackage{listings}
%\usepackage{subfig}

\usetheme{Copenhagen}
\usecolortheme{seahorse}
 
 
%Information to be included in the title page:
\title{Queue Data Structure}
\author{Vivek K. S., Deepak G.}
\institute{Information Systems Decision Sciences (ISDS)\\
MUMA College of Business\\
University of South Florida \\
Tampa, Florida}
\date{2017}
 
 
 
\begin{document}
 
\frame{\titlepage}
 
\begin{frame}
\frametitle{What is a Queue}
\begin{itemize}

\item A queue is an ordered collection of items.

\item Data items are added through one end called the “rear” and removed through the other called the “front”.


\item It follows the FIFO ordering principle also known as "first-come first-served.”
 

\item Ticket counters and Printing tasks in a library are real world examples of a queue.


\end{itemize}
\end{frame}


\begin{frame}
\frametitle{Essential Operations in a Queue}
\begin{itemize}

\item Enqueue - Ability to add a new item to the queue.

\item Dequeue - Ability to remove an item from the queue.

\item Ability to check if the queue is empty.

\item Ability to check the size of the queue.


\end{itemize}
\end{frame}

\begin{frame}
\frametitle{Logical Approach to Implementing a Queue}
\begin{itemize}

\item We need to be able to create new queue instances on the go. Hence we will take an object oriented approach towards building a Queue and defining its behavior.

\item The list data structure provides us with the methods to perform all these operations.

\item Enqueuing can be done from the rear end using insert() method at index position 0 so the insertion operation will be a O(1) operation.

\item Dequeuing can be done from the front end using the pop() method in lists.

\end{itemize}
\end{frame}


\begin{frame}
\frametitle{Continued..}
\begin{itemize}



\item The comparator operator will return a True/False based on the fact that the queue is empty or not.

\item The len() method can be used to check the size of the queue.



\end{itemize}
\end{frame}

\begin{frame}[fragile]
\frametitle{Python code}

Code Implementation in Python
\begin{lstlisting}[language=Python]

class Queue:
	def __init__(self):
		self.items = []
	def is_empty(self):
		return self.items == []
	def add(self, item):
		self.items.insert(0,item)
	def remove(self):
		return self.items.pop()
	def length(self):
		return len(self.items)
\end{lstlisting}

\end{frame}

\begin{frame}
\frametitle{Summary}
\begin{itemize}
\item Queues are abstract data structures that can be built using a List type.
\item It follows the FIFO ordering principle and finds its application in most real world applications that follow the FIFO ordering principle, for example a printing task in a lab.
\item Queues are very useful in most computing applications and the right implementation gives the best performance, as seen in the enqueue operation which is O(1).
\end{itemize}
\end{frame}
\end{document}


\documentclass{beamer}
\usepackage[utf8]{inputenc}
\usepackage{graphicx}
\usepackage{subcaption}
\usepackage{listings}
%\usepackage{subfig}

\usetheme{Madrid}
\usecolortheme{seahorse}
\usefonttheme{serif} 
 
 
%Information to be included in the title page:
\title{The Ordered List Data Structure}
\author{Vivek K. S., Deepak G.}
\institute{Information Systems Decision Sciences (ISDS)\\
MUMA College of Business\\
University of South Florida \\
Tampa, Florida}
\date{2017}
 

\begin{document}
 
\frame{\titlepage}
 

\begin{frame}
\frametitle{The Ordered List}
\begin{itemize}

\item An ordered list is another implementation of the List data structure.

\item In this type, the items of the list maintain a relative position to each other, in that they are ordered.

\item The ordering is typically ascending or descending based on the choice of implementation.

\item The list items must have a meaningful comparison operation in place.

\item Many of the operations of the ordered list are the same as the unordered list where the ordered characteristic is not involved.

\end{itemize}
\end{frame}
\begin{frame}
\frametitle{Essential operations in an Ordered List}
\begin{itemize}

\item Ability to create List instances.
\item Ability to add items to the list while also making sure that the ordering is preserved. We assume the item is not already present in the list.

\item Ability to remove an item from the list.
\item Searching and finding an item.
\item Ability to check if the list is empty or not.
\item Find out the size of the list.

\item Removing the last item in the list by popping it.

\item Returning the index position of a list item.
\end{itemize}
\end{frame}


\begin{frame}
\frametitle{Implementing an Ordered List}
In order to implement the ordered list, we must follow certain rules such as, 

\begin{itemize}
\item Keep the ordered structure intact.
\item Use the linked structure and notion of Node to achieve the relative positioning of the items.
\item Like in the unordered list, once again, an empty list will be denoted by a head reference to None.
\item All the operations, except the search and add operations are the same as seen in unordered list.
\end{itemize}
\end{frame}

\begin{frame}
\frametitle{The Add operation}
The add method in ordered lists differ very much from the add method in the Unordered list.

\begin{itemize}
\item Using Link traversal, we must traverse the list to find out where the item could be inserted depending on the order.

\item For example, in our list containing the following elements [17,23,45,68,78,79,90], if we have to add 75, the method should know that it goes in between 68 and 78.

\item It is therefore helpful to perform this operation with an additional reference along with the current reference as seen in previous exercises. 

\end{itemize}
\end{frame}
\begin{frame}[fragile]
\begin{lstlisting}[language=Python]

def add(self, item):
	current = self.head
	tail = None
	stop = False
	while current != None and not stop:
 		if current.get_data() > item:
 			stop = True
 		else:
			tail = current
			current = current.get_next()

	temp = Node(item)
	if tail == None:
		temp.set_next(self.head)
		self.head = temp
	else:
		temp.set_next(current)
		tail.set_next(temp)
\end{lstlisting}
\end{frame}

\begin{frame}
\frametitle{Search Operation}

The search operation works pretty much similar to a search in unordered list.

\begin{itemize}
\item The same link traversal mechanism is used, but since, the items are sorted, the  operation is more efficient here.

\item For example, if in a list containing [17,23,45,68,78,79,90], we are looking for 48.

\item When the links are traversed in search of the item, after 45 is passed, when the next node with value 68 is encountered, the comparison will result in a value greater than the item being searched for and therefore the search will be terminated right there.

\item Because, the underlying characteristic of the list has already told us that going further is inconsequential. 
\end{itemize}
\end{frame}

\begin{frame}[fragile]
\begin{lstlisting}[language=Python]
def search(self, item):
	current = self.head
	found = False
	stop = False
	while current != None and not found and
	 	not stop:
		if current.get_data() == item:
			found = True
		else:
			if current.get_data() > item:
				stop = True
			else:
				current = current.
				get_next()
	return found
\end{lstlisting}
\end{frame}


\begin{frame}
\frametitle{Summary}
\begin{itemize}
\item Ordered lists are more or less similar to unordered lists except a few key operations such as  add and search due to the unique underlying structure.
\item Ordered lists offer significant improvements in search over the unordered counterparts because the items are ordered and mostly the average case performance is significantly higher.
\item Most operations are similar  in implementation and performance to the unordered lists.
\end{itemize}
\end{frame}
\end{document}


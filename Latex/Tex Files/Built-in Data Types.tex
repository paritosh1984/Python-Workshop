\documentclass{beamer}
\usepackage[utf8]{inputenc}
\usepackage{graphicx}
\usepackage{subcaption}
\usepackage{listings}
%\usepackage{subfig}
\usetheme{Copenhagen}
\usecolortheme{seahorse}
 
 
%Information to be included in the title page:
\title{Built-in Data Types in Python}
\author{Vivek K. S., Deepak G.}
\institute{Information Systems Decision Sciences (ISDS)\\
MUMA College of Business\\
University of South Florida \\
Tampa, Florida}
\date{2017}
 
 
 
\begin{document}
 
\frame{\titlepage}
 
\begin{frame}
\frametitle{Objectives}

\begin{itemize}
\item To understand built-in data types such as integers, floats, strings and Boolean type.
\item To learn their usage and specific behavior.
\item To understand objects in Python.
\item To understand dynamic typing in Python type conversion.
\item To understand variables and references.
\end{itemize}
\end{frame}

\begin{frame}
\frametitle{Objects in Python}
\begin{itemize}
\item In Python, every data structure is an object. 
\item Integers, Strings, Lists, Functions, Modules are all implemented as objects. 
\item An object is in its simple definition a  block(s) of memory that contains data in it.
\item The data has a type associated with it and consequentially a set of behaviors and what could be done on/with it.
\item The type of data also determines its mutability and immutability.
\item The advantage of this implementation is consistency.
\end{itemize}
\end{frame}

\begin{frame}[fragile]
\frametitle{Dynamic Typing in Python}
\begin{itemize}
\item Python is a dynamically typed language.
\item Unlike statically-typed languages like Java or C++, Python does not attach the type of an object to its variable identifier.
\item Instead, the assignment of an object to an identifier simply attaches a name to the block of memory containing the data and acts only a reference to it. 
\item In python, we use the type() method to identify the type of the variable.
\end{itemize}
\begin{lstlisting}[language=Python, keywordstyle=\color{blue}]
	# Code to identify data type
	a = 10
	print(a)
	type(a)


\end{lstlisting}
\end{frame}

\begin{frame}[fragile]
\frametitle{Rules for Identifier Names}
The following are some of the rules associated with identifier names in Python.
The following characters are allowed in identifier names.
\begin{itemize}
\item Lowercase letters (a through z)
\item Uppercase letters (A through Z)
\item Digits (0 through 9)
\item \begin{lstlisting}[language=Python]
Underscore (_)
\end{lstlisting}
\end{itemize}
Their usage is as follows.
\begin{itemize}
\item Names cannot begin with a digit as seen in most programming languages. 
\item Python treats names that begin with an underscore in special ways.
\item Reserved keywords in Python cannot be used for identifier names.
\end{itemize}
\end{frame}
\end{document}

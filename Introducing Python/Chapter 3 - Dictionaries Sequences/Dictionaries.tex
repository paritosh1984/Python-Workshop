\documentclass{beamer}
\usepackage[utf8]{inputenc}
\usepackage{graphicx}
\usepackage{subcaption}
\usepackage{listings}
%\usepackage{subfig}
\usetheme{Copenhagen}
\usecolortheme{seahorse}
 
 
%Information to be included in the title page:
\title{Dictionaries in Python}
\author{Vivek K. S., Deepak G.}
\institute{Information Systems Decision Sciences (ISDS)\\
MUMA College of Business\\
University of South Florida \\
Tampa, Florida}
\date{2017}
 
\expandafter\def\expandafter\insertshorttitle\expandafter{%
\insertshorttitle\hfill%
\insertframenumber\,/\,\inserttotalframenumber}

\lstset{language=python,
		showstringspaces=false,
                basicstyle=\ttfamily,
                keywordstyle=\color{blue}\ttfamily,
                stringstyle=\color{red}\ttfamily,
                commentstyle=\color{purple}\ttfamily,
                morecomment=[l][\color{magenta}]{\#}
}

\begin{document}
\frame{\titlepage}

\begin{frame}
\frametitle{Dictionaries}
\begin{itemize}
\item A dictionary is similar to a list, in that it is a collection of items.
\item Unlike lists, the order of the items doesn't matter.
\item Dictionary elements are not identified by offset indicing as seen in lists (0,1,2..).
\item Instead, in dictionaries, the items are \textit{key-value} pairs and they are uniquely identified by the keys as they are unique. 
\item The keys are usually of String type, but they can be any immutable type in Python as integers, booleans, floats and tuples.
\item Dictionaries are mutable.
\end{itemize}
\end{frame}

\begin{frame}[fragile]
\frametitle{Creating a Dictionary}
Dictionaries can be created as follows:
\begin{lstlisting}
# Creating an empty dictionary
empty_dict = {}
print(empty_dict)
{}
# Creating a dictionary with key-value pairs.
capitals={'New York':'Albany',
'California':'Sacramento'}
print(capitals)
{'California': 'Sacramento', 
'New York': 'Albany'}
\end{lstlisting}
\end{frame}

\begin{frame}[fragile]
\frametitle{Converting other Sequences into Dictionaries}
Other Sequence Types can be converted to Dictionary objects as follows:
\begin{lstlisting}
# A list of two-item tuples:
my_tuple = [ ('a', 'b'), ('c', 'd'), 
('e', 'f') ]
my_dict = dict(my_tuple)
print(my_dict)
{'a': 'b', 'c': 'd', 'e': 'f'}

# A list of two-item lists:
my_list = ( ['a', 'b'], ['c', 'd'], ['e', 'f'] )
new_dict = dict(my_list)
print(new_dict)
{'c': 'd', 'a': 'b', 'e': 'f'}
\end{lstlisting}
\end{frame}

\begin{frame}[fragile]
\frametitle{Continued..}
\begin{lstlisting}
# A list of two-character strings:
my_str_list = [ 'ab', 'cd', 'ef' ]
my_dict = dict(my_str_list)
my_dict
{'a': 'b', 'c': 'd', 'e': 'f'}

# A tuple of two-character strings:
my_str_tuple = ( 'ab', 'cd', 'ef' )
my_dict = dict(my_str_tuple)
my_dict
{'a': 'b', 'c': 'd', 'e': 'f'}
\end{lstlisting}
It is to be noted that this works only for two-item sequences. For more than 2, an exception will be thrown.
\end{frame}

\begin{frame}
\frametitle{Operations in Dictionaries}
\begin{itemize}
\item Adding and modifying key-value pairs.
\item Combine Dictionaries with Update().
\item Deleting items.
\item Deleting all items using clear().
\item Membership test.
\item Fetching items from the dictionary.
\item Assigning using assignment sign and by copying.
\end{itemize}
\end{frame}

\begin{frame}
\frametitle{More Examples in Dictionaries}
More code examples are available at --
 
\url{https://github.com/vivek14632/Python-Workshop/tree/master/Introducing\%20Python/Chapter\%203}
\end{frame}

\begin{frame}
\frametitle{Exercise}
Create a dictionary with your name, address and fetch your zip code from the dictionary object.
\end{frame}

\begin{frame}
\frametitle{Summary}
\begin{itemize}
\item We learned a very important sequence data structures in Python, the Dictionary type.
\item We learned the different capabilities it provides and how it differ from Lists and Tuples.
\item We learned the different operations that could be performed on/with Dictionaries.
\end{itemize}
\end{frame}
\end{document}
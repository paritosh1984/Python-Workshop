\documentclass{beamer}
\usepackage[utf8]{inputenc}
\usepackage{graphicx}
\usepackage{subcaption}
\usepackage{listings}
%\usepackage{subfig}
\usetheme{Copenhagen}
\usecolortheme{seahorse}
 

%Information to be included in the title page:
\title{Structural Text Files}
\author{Vivek K. S., Deepak G.}
\institute{Information Systems Decision Sciences (ISDS)\\
MUMA College of Business\\
University of South Florida \\
Tampa, Florida}
\date{2017}
 
\begin{document}
\frame{\titlepage}

\begin{frame}
\frametitle{Structured Text Files}
\begin{itemize}
\item With flat files (simple text files), one level of organizing data is the line.
\item But in most real-world scenarios, we will require more structure than that. We will require to save the data for future reuse and even exchange with other programs.
\item There are many formats
\begin{itemize}
\item A separator, or delimiter, character like tab ('\t'), comma (','), or vertical bar
('|'). CSV files are an example of this format.
\item '<' and '>' around tags. Examples include XML and HTML.
\end{itemize}
\end{itemize}
\end{frame}

\begin{frame}
\frametitle{CSV File}
\begin{itemize}
\item Delimited files are often used as an exchange format for spreadsheets and databases.
\item CSV files could be read one line at a time splitting each line into fields at
comma separators, and adding the results to data structures such as lists and dictionaries.
But its better to use the standard csv module, because parsing these files can get
more complicated than you think.
\end{itemize}
\end{frame}
\end{document}

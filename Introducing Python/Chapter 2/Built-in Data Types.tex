\documentclass{beamer}
\usepackage[utf8]{inputenc}
\usepackage{graphicx}
\usepackage{subcaption}
\usepackage{listings}
%\usepackage{subfig}
\usetheme{Copenhagen}
\usecolortheme{seahorse}
 
 
%Information to be included in the title page:
\title{Built-in Data Types in Python}
\author{Vivek K. S., Deepak G.}
\institute{Information Systems Decision Sciences (ISDS)\\
MUMA College of Business\\
University of South Florida \\
Tampa, Florida}
\date{2017}
 
\expandafter\def\expandafter\insertshorttitle\expandafter{%
\insertshorttitle\hfill%
\insertframenumber\,/\,\inserttotalframenumber}

\lstset{language=python,
		showstringspaces=false,
                basicstyle=\ttfamily,
                keywordstyle=\color{blue}\ttfamily,
                stringstyle=\color{red}\ttfamily,
                commentstyle=\color{purple}\ttfamily,
                morecomment=[l][\color{magenta}]{\#}
}
 
\begin{document}
 
\frame{\titlepage}
 
\begin{frame}
\frametitle{Outline}

\begin{itemize}
\item To understand objects in Python.
\item To understand built-in data types such as integers, floats, strings and boolean type.
\item To learn their usage and specific behavior.
\item To understand dynamic typing in Python.
\item To understand variables and references.
\end{itemize}
\end{frame}

\begin{frame}
\frametitle{Objects in Python}
\begin{itemize}
\item In Python, everything is an object.
\item Different programming languages have different definitions to objects.
\item In some languages such as Java or C++, it is a very strict definition that dictates that all objects must have methods and attributes associated with them. Such objects are subclassable (inheritable).
\item In Python, the definition is a little loose, which says that objects may or may not have attributes and methods.
\item Everything that can be assigned to a variable and passed as an argument to a function qualifies to be an object in Python.
\end{itemize}
\end{frame}

\begin{frame}
\frametitle{Objects in Python}
From Guido's Blog --

\textit{One of my goals for Python was to make it so that all objects were "first class." By this, I meant that I wanted all objects that could be named in the language (e.g., integers, strings, functions, classes, modules, methods, etc.) to have equal status. That is, they can be assigned to variables, placed in lists, stored in dictionaries, passed as arguments, and so forth.}
\end{frame}

\begin{frame}
\frametitle{Objects in Python}
\begin{itemize}
\item Integers, Strings, Lists, Functions, Modules are all implemented as objects. 
\item An object is in its simple definition a  block(s) of memory that contains data in it.
\item The data has a type associated with it and consequently a set of behaviors (methods) and what could be done with them to change its attributes.
\item The type of data also determines its mutability and immutability.
\item The advantage of this implementation is consistency.
\end{itemize}
\end{frame}

\begin{frame}[fragile]
\frametitle{Dynamic Typing in Python}
\begin{itemize}
\item Python is a dynamically typed language.
\item Unlike statically-typed languages such as Java or C++, Python does not attach the type of an object to its variable identifier.
\item The assignment of an object to an identifier in Python simply attaches a name to the block of memory containing the data and acts only as a reference to it. 
\item In Python, we use the type() method to identify the type of the variable.
\end{itemize}
\begin{lstlisting}[language=Python, keywordstyle=\color{blue}]
	# Code to identify data type
	a = 10
	print(a)
	type(a)


\end{lstlisting}
\end{frame}

\begin{frame}[fragile]
\frametitle{Rules for Identifier Names}
The following are some of the rules associated with identifier names in Python.
The following characters are allowed in identifier names.
\begin{itemize}
\item Lowercase letters (a through z)
\item Uppercase letters (A through Z)
\item Digits (0 through 9)
\item Underscore (\_)
\end{itemize}
Their usage is as follows:
\begin{itemize}
\item Names cannot begin with a digit.
\item Python treats names that begin with an underscore in special ways. For example \_\_init\_\_ 
\item Reserved keywords in Python cannot be used for identifier names. \url{https://docs.python.org/2.5/ref/keywords.html}
\end{itemize}
\footnote{More info on use of underscore at \url{https://hackernoon.com/understanding-the-underscore-of-python-309d1a029edc}}
\end{frame}

\begin{frame}
\frametitle{Built-in Types in Python}
The following types are the basic  building blocks of programming in Python. Dealing with numbers and textual data is the cornerstone of any programming language and it is no different in Python.

Python has in-built support for,
\begin{itemize}
\item Numbers
\begin{itemize}
\item Integers (whole numbers such as 5 and 1,000,000,000).
\item Floating point numbers (such as 3.1416, 14.99, and 1.87e4).
\end{itemize}
\item Strings (The first Python Sequence Type. Python 3 supports Unicode standard).
\item Boolean Types (That represents the two states True and False).
\end{itemize}
\end{frame}


\begin{frame}[fragile]
\frametitle{Unique Operations in Python}
Python offers some unique flavors in the most common operations we use on a daily basis:

It offers two types of division operation.
\begin{lstlisting}[language=Python]
# '/' (slash) carries out decimal division.
7/2 = 3.5.
# '//' (double slash) carries out integer
# division also called as floor division.
7//2 = 3

x = 77
x //= 10
print(x)
7
\end{lstlisting}
\end{frame}

\begin{frame}[fragile]
\frametitle{Arithmetic Operation and Assignment in Python}
An arithmetic operation and assignment of the result could be done as easy as follows between two operands.

\begin{lstlisting}[language=Python]
Adding two integers (numbers).
a = 95
a -= 3
print(a)
92

a *= 2
print(a)
184
\end{lstlisting}
\end{frame}

\begin{frame}[fragile]
\frametitle{Common Operations in Python}
The Modulo operator returns the remainder of a division operation.
\begin{lstlisting}[language=Python]
Modulo operation 9%5 = 4

# Getting the reminder & quotient can also 
be done by using divmod()

divmod(9,5) = (1,4)

\end{lstlisting}
\end{frame}

\begin{frame}[fragile]
\frametitle{Type Conversions in Python}
Python also offers ways to convert one type into another:

\begin{lstlisting}[language=Python]
# Converting a Boolean to integer
>>>int(True)
1
>>>int(False)
0

# Converting a Float to integer

>>>int(55.5)
55
>>>int(1.0e4)
10000

\end{lstlisting}
\end{frame}
\begin{frame}[fragile]
\frametitle{Continued..}
\begin{lstlisting}[language=Python]
# Converting a String to integer
>>>int('99')
99
>>>int('-55')	
-55

# Converting an int to char
>>>chr(97)
'a'
Getting the ASCII code of a character.
>>>ord('a')
97

ord() and chr() are built-in functions in Python.
 
\end{lstlisting}
\footnote{\tiny{Refer \url{https://docs.python.org/3/library/functions.html} for more details on built-in functions.}}.
\end{frame}

\begin{frame}[fragile]
\frametitle{Integer Overflow}
\begin{itemize}
\item Python handles really long numbers with ease.
\item This is a feature which most Programming languages have a problem with, commonly referred to as "Integer overflow".
\item For example 10**100 (10 raised to the power 100) will result in a huge number called the "Googol".
\item Even if an multiplication is to be performed between two Googols, Python can easily handle that math and support the operation.
\item Lets try it.
\end{itemize}
\begin{lstlisting}[language=Python]
	googol = 10**100
	print(googol*googol)
\end{lstlisting}
\end{frame}

\begin{frame}
\frametitle{Floats in Python}
Floats are numbers with decimal points. 
\begin{itemize}
\item All the operations that work on integers can be applied to floats as well.
\item Float type conversion can be done using the method float() with the data passed as an argument to it.
\item Strings can be converted to floats as well.

\end{itemize}
\end{frame}

\begin{frame}[fragile]
\frametitle{Strings in Python}
A String is a sequence type in Python. It is simply a sequence of characters.
\begin{itemize}
\item In Python, Strings are immutable.
\item The data in a String cannot be modified, but copies can be created.
\item Strings in Python 3 support unicode operations.
\item In Python, Strings can be written using both single and double quotes, which allows  double quotes to be written within single quotes and vice versa.

\end{itemize}
\begin{lstlisting}[language=Python]
>>>'Python'
Python
>>>"Python"
Python
\end{lstlisting}
\end{frame}

\begin{frame}[fragile]
\frametitle{Triple Quotes and Docstrings}
Triple Quotes is a unique flavor offered by Python.
\begin{itemize}
\item They are most commonly used to create multi-line Strings that are sometimes used as comments and docstrings.
\item A docstring is a string literal specified in a function or any piece of code as a comment, to document that specific segment of code. For example, to add a docstring to a function definition,
 
\end{itemize}

\begin{lstlisting}[language=Python]
def add(x,y):
	'''The function adds two integers
	and returns the sum.'''
\end{lstlisting}
\end{frame}

\begin{frame}[fragile]
\frametitle{Continued..}
There are two ways to show the docstring.

\begin{lstlisting}
# Method #1 -Using help()

help(add)

'Help on function add in module __main__:

add(x, y)
The function adds two integers and returns 
the sum.'

# Method #2 - Using __doc__	
add.__doc__

'The function adds two integers and 
returns the sum.'
\end{lstlisting}
\end{frame}

\begin{frame}
\frametitle{Common String Operations}
The following are some of the common String operations.
\begin{itemize}
\item String Concatenation.
\item Duplication.
\item Replacing a substring.
\item Indexing and Slicing.
\item Finding the length.
\item Splitting and Stripping.
\item String formatting operations.
\end{itemize}
\end{frame}

\begin{frame}
\frametitle{More Examples}
More code examples are available at --
 
\url{https://github.com/vivek14632/Python-Workshop/tree/master/Introducing\%20Python/Chapter\%202}
\end{frame}

\begin{frame}
\frametitle{Exercise}
\begin{itemize}
\item Problem \#1
\begin{itemize}
\item Create a function to add two numbers and also add documentation to it.
\item Further, try to retrieve the documentation.
\end{itemize}
\item Problem \#2
\begin{itemize}
\item Write a code to find area of a circle whose radius is 5 inches.
\end{itemize}
\end{itemize}
\end{frame}
\begin{frame}
\frametitle{Summary}
\begin{itemize}
\item We understood Built-in data types in Python.
\item We learned how Python works as a Dynamically typed language.
\item We understood the use of integers, strings and floats and the operations that could be performed on them.
\item We understood type conversion between different types.
\end{itemize}
\end{frame}
\end{document}

\documentclass{beamer}
\usepackage[utf8]{inputenc}
\usepackage{graphicx}
\usepackage{subcaption}
\usepackage{listings}
%\usepackage{subfig}
\usetheme{Copenhagen}
\usecolortheme{seahorse}
 
 
%Information to be included in the title page:
\title{Built-in Data Types in Python}
\author{Vivek K. S., Deepak G.}
\institute{Information Systems Decision Sciences (ISDS)\\
MUMA College of Business\\
University of South Florida \\
Tampa, Florida}
\date{2017}
 
 
 
\begin{document}
 
\frame{\titlepage}
 
\begin{frame}
\frametitle{Objectives}

\begin{itemize}
\item To understand built-in data types such as integers, floats, strings and Boolean type.
\item To learn their usage and specific behavior.
\item To understand objects in Python.
\item To understand dynamic typing in Python type conversion.
\item To understand variables and references.
\end{itemize}
\end{frame}

\begin{frame}
\frametitle{Objects in Python}
\begin{itemize}
\item In Python, every data structure is an object. 
\item Integers, Strings, Lists, Functions, Modules are all implemented as objects. 
\item An object is in its simple definition a  block(s) of memory that contains data in it.
\item The data has a type associated with it and consequentially a set of behaviors and what could be done on/with it.
\item The type of data also determines its mutability and immutability.
\item The advantage of this implementation is consistency.
\end{itemize}
\end{frame}

\begin{frame}[fragile]
\frametitle{Dynamic Typing in Python}
\begin{itemize}
\item Python is a dynamically typed language.
\item Unlike statically-typed languages like Java or C++, Python does not attach the type of an object to its variable identifier.
\item Instead, the assignment of an object to an identifier simply attaches a name to the block of memory containing the data and acts only a reference to it. 
\item In python, we use the type() method to identify the type of the variable.
\end{itemize}
\begin{lstlisting}[language=Python, keywordstyle=\color{blue}]
	# Code to identify data type
	a = 10
	print(a)
	type(a)


\end{lstlisting}
\end{frame}

\begin{frame}[fragile]
\frametitle{Rules for Identifier Names}
The following are some of the rules associated with identifier names in Python.
The following characters are allowed in identifier names.
\begin{itemize}
\item Lowercase letters (a through z)
\item Uppercase letters (A through Z)
\item Digits (0 through 9)
\item \begin{lstlisting}[language=Python]
Underscore (_)
\end{lstlisting}
\end{itemize}
Their usage is as follows.
\begin{itemize}
\item Names cannot begin with a digit as seen in most programming languages. 
\item Python treats names that begin with an underscore in special ways.
\item Reserved keywords in Python cannot be used for identifier names.
\end{itemize}
\end{frame}

\begin{frame}[fragile]
\frametitle{Unique Operations in Python}
Python offers some unique flavors in the most common operations we use on a daily basis:

Python offer two types of division operation.
\begin{lstlisting}[language=Python]
/ carries out decimal division.
7/2 = 3.5.
// carries out integer division also called 
as floor division.
7//2 = 3

x = 77
x //= 10
x
[Output] 7
\end{lstlisting}
\end{frame}

\begin{frame}[fragile]
\frametitle{Assignment and Operation in Python}
Python offers some unique flavors in the most common operations we use on a daily basis:

Python offer two types of division operation.
\begin{lstlisting}[language=Python]
Adding two integers (numbers) could be done as simple as 
a = 95
a -= 3
a
[Output] 92
a *= 2
a
[Output] 184
\end{lstlisting}
\end{frame}

\begin{frame}[fragile]
\frametitle{Other Operations in Python}
\begin{lstlisting}[language=Python]
Modulo operation 9%5 = 4

Getting the reminder & quotient can 
be done by using divmod()

divmod(9,5) = (1,4)

\end{lstlisting}
\end{frame}

\begin{frame}[fragile]
\frametitle{Type Conversions in Python}
Python offers way to convert one type into another:

\begin{lstlisting}[language=Python]
Converting a Boolean to int

>>>int(True)
1
>>>int(False)
0

Converting a Float to int

>>>int(55.5)
55
>>>int(1.0e4)
10000

\end{lstlisting}
\end{frame}
\begin{frame}[fragile]
\frametitle{Continued..}
Python offers way to convert one type into another:

\begin{lstlisting}[language=Python]
Converting a String to int
>>>int('99')
99
>>>int('-55')	
-55

Converting an int to char
>>>chr(97)
'a'
Getting the ASCII code of a character.
>>>ord('a')
97

ord() and chr() are built-in functions in Python.

footnote - Refer https://docs.python.org/3/library/functions.html for more details.
\end{lstlisting}
\end{frame}

\begin{frame}
\frametitle{Integer Overflow}
\begin{itemize}
\item Python handles really long numbers with ease.
\item This is a feature which most Programming languages have a problem with commonly referred to a s "Integer overflow".
\item For example 10**100 (10 raised to the power 100) will result in a huge number called the googol.
\item Even if an multiplication is to be performed between two googols, Python can easily handle that math and support the operation.
\item Lets try it.
\end{itemize}
\end{frame}

\begin{frame}
\frametitle{Floats in Python}
Floats are numbers with decimal points. 
\begin{itemize}
\item All the operations that work on integers can be applied to floats as well.
\item Float type conversion can be done using float()
\item Strings can be converted to floats as well.

\end{itemize}
\end{frame}

\begin{frame}[fragile]
\frametitle{Strings in Python}
A String is a sequence type in Python. It is basically a string of characters.
\begin{itemize}
\item In Python, Strings are immutable.
\item The data in a String cannot be modified, but copies can be created.
\item Strings in Python 3 support unicode operations.
\item In Python, Strings can be written using both single and double quotes, which allows  double quotes to be written within single quotes and vice versa.

\end{itemize}
\begin{lstlisting}[language=Python]
>>>'Python'
Python
>>>"Python"
Python
\end{lstlisting}
\end{frame}

\begin{frame}[fragile]
\frametitle{Triple Quotes and Docstrings}
Triple Quotes is a unique flavor offered by Python.
\begin{itemize}
\item They are most commonly used to create multi-line Strings.
\item One of their most common uses is in creating docstrings.
\item A docstring is a string literal specified in a function or any piece of code as a comment, to document that specific segment of code.
 
\end{itemize}

\begin{lstlisting}[language=Python]
	def add(x,y):
		'''The function add two integers
		and returns the sum.'''
\end{lstlisting}
\end{frame}

\begin{frame}[fragile]
\frametitle{Continued..}
There are two ways to see the docstring.

\begin{lstlisting}[language=Python]
	def add(x,y):
		'''The function add two integers
		and returns the sum.'''
	 
	help(add)
	Help on function add in module __main__:

	add(x, y)
    	The function adds two integers 
    	and returns the sum.
    	
    add.__doc__
    'The function adds two integers \n and 
    returns the sum.'
\end{lstlisting}
\end{frame}

\begin{frame}
\frametitle{Common String Operations}
The following are some of the common operations in a String.
\begin{itemize}
\item String Concatenation.
\item Duplication.
\item Replacing a substring.
\item Indexing and Slicing.
\item Finding the length.
\item Splitting and Stripping.
\item String formatting operations.
\end{itemize}
\end{frame}

\begin{frame}
\frametitle{Summary}
\begin{itemize}
\item We understood Build-in data types in Python.
\item We learned how Python works as a Dynamically typed language.
\item We understood the use of integers, strings and floats and the operation that could be performed on them.
\item We learned how these data types act as a building block towards building larger complex program structures. 
\end{itemize}
\end{frame}
\end{document}

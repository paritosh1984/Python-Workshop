\documentclass{beamer}
\usepackage[utf8]{inputenc}
\usepackage{graphicx}
\usepackage{subcaption}
\usepackage{listings}
%\usepackage{subfig}
\usetheme{Copenhagen}
\usecolortheme{seahorse}
 

%Information to be included in the title page:
\title{Modules, Packages and Programs}
\author{Vivek K. S., Deepak G.}
\institute{Information Systems Decision Sciences (ISDS)\\
MUMA College of Business\\
University of South Florida \\
Tampa, Florida}
\date{2017}

\expandafter\def\expandafter\insertshorttitle\expandafter{%
\insertshorttitle\hfill%
\insertframenumber\,/\,\inserttotalframenumber}

\lstset{language=python,
		showstringspaces=false,
                basicstyle=\ttfamily,
                keywordstyle=\color{blue}\ttfamily,
                stringstyle=\color{red}\ttfamily,
                commentstyle=\color{purple}\ttfamily,
                morecomment=[l][\color{magenta}]{\#}
}

\begin{document}
\frame{\titlepage}

\begin{frame}
\frametitle{Objectives}
\begin{itemize}
\item To learn about packages, modules and programs.
\item To understand the reusability of Python Packages and how they could be imported  into another program.
\item To learn how to create packages and modules.
\item To learn how to create standalone programs and run them from the interactive prompt.
\item To understand how to move away from interactive programming to creating standalone programs with function calls, namespaces and deriving the the desired output.
\end{itemize}
\end{frame}


\begin{frame}[fragile]
\frametitle{First Standalone Program}
\begin{itemize}
\item Create a file called ex1.py on your computer. Open it in edit mode.
\item Write the following piece of code.
\begin{lstlisting}
print(5+5)
\end{lstlisting}
\item The line should print the sum 10 on the screen when it is executed.
\item To run the file, we'll use the Command  prompt (terminal for Mac users).
\item At the prompt, we'll say python ex1.py and hit return.
\item The number "10" should be printed on the screen.
\end{itemize}
\end{frame}

\begin{frame}[fragile]
\frametitle{Command Line Arguments}
\begin{itemize}
\item Arguments can be passed to our program directly from the command line as well.
\item Lets create a second file called ex2.py and write the following line of code in it.
\begin{lstlisting}
import sys
print('Program arguments : ', sys.argv)
print(type(sys.argv))
\end{lstlisting}
\item Execute the code from the command prompt as "python ex2.py".
\item You will notice that the name of the python file gets printed on the screen. It is because, the .py file is an argument in itself.
\item Passing a couple of more arguments, we can understand that the arguments are internally converted into a list and the first argument to the program is always the file name itself. 
\end{itemize}
\end{frame}

\begin{frame}[fragile]
\frametitle{Importing Modules}
\begin{itemize}
\item A module is a Python code file that can be reused.
\item Modules are reused by simply importing them into the current program. Example:
\begin{lstlisting}[language=Python]
dice.py
def throw_dice():
	"""Roll your dice like a player!"""
	from random import choice
	possibilities = [1,2,3,4,5,6]
	return choice(possibilities)

ex3.py
	import dice
	side = dice.throw_dice()
	print("You have rolled a ",side)
\end{lstlisting}
\end{itemize}
\end{frame}

\begin{frame}[fragile]
\frametitle{Explanation}
\begin{itemize}
\item In the example, we see two types of import statements.
\item In dice.py, we import only the choice function from the random module. This is usually a best practice as it saves on memory and avoids unnecessary code imports.
\item The random module is a fully standalone python code  with multiple functions and \textit{choice} is one amongst them.
\item In the ex3.py file, we import the dice.py module and using dice as a handle, we call the throw\_dice() function of it.
\item Remember that throw\_dice() method is a property  of the dice object (In python, everything is treated as an object).
\item The dot notation is used to call the property as seen in dice.throw\_dice().
\end{itemize}
\end{frame}

\begin{frame}[fragile]
\frametitle{Using Aliases for Module Names}
\begin{itemize}
\item Modules can be imported with an alias name as well.
\item The new alias name can then be used to access any property of the imported module and this results in ease of typing.
Example,
\begin{lstlisting}[language=Python]
import numpy as np
arr = np.array([1,2,3,4,5])
print(arr.shape)
\end{lstlisting}
\item In the code, we assigned an alias name to the numpy module as 'np'.
\item It is later used to call the array function of the module as "np.array()".
\end{itemize}
\end{frame}


\begin{frame}[fragile]
\frametitle{Packages}
\begin{itemize}
\item Modules can be further organized into file hierarchies called Packages.
\item Maybe in the previous exercise, we may want to roll one die and sometimes roll two dice.
\item One way to structure this is to have two separate modules, one for a single die and another for a pair of dice and each of them will have a function called roll.
\item It will look something like this. 
\begin{lstlisting}[language=Python]

# single.py
def roll_dice():
	"""Roll your dice like a player!"""
	from random import choice
	possibilities = [1,2,3,4,5,6]
	return choice(possibilities)
\end{lstlisting}

\end{itemize}
\end{frame}

\begin{frame}[fragile]
\frametitle{Packages Continued}
 
\begin{lstlisting}[language=Python]

#double.py
def roll_dice():
	"""Roll a pair of dice"""
	from random import choice
	possibilites = [1,2,3,4,5,6]
	return(choice([(one, two) 
	for one in possibilites 
	for two in possibilites]))
\end{lstlisting}
We'll add another python file called \_\_init\_\_.py that contains the following code.
\begin{lstlisting}
#__init__.py
from roll import single
from roll import double

\end{lstlisting}

\end{frame}

\begin{frame}[fragile]
\frametitle{Packages}
 
\begin{lstlisting}[language=Python]

#ex5.py

from roll import single,double

side = single.roll_dice()
print("You have rolled a {}".format(side))

sides = double.roll_dice()
print("You have rolled {} and {}".
format(sides[0],sides[1]))
\end{lstlisting}

When the above code is executed, the modules "single" and "double" will be imported from the package "roll".	

\end{frame}

\begin{frame}
\frametitle{Explanation}
\begin{itemize}
\item The way the package is created is by placing all our modules in a single directory that has been named with the desired Package's name.
\item The next step is to place a python file called \_\_init\_\_.py for Python to identify the directory as a Package.
\item Once that is done, we are free to import the entire package or just the select modules from the package into our program using \textit{from} as follows \textit{from roll import single}.
\item The methods of the module can be called by the module name or the name(alias) assigned to the module as part of the import.
\end{itemize}
\end{frame}

\begin{frame}[fragile]
\frametitle{The \_\_init\_\_.py file}
\begin{itemize}
\item The \_\_init\_\_.py file is required if we import using * shown below.
\end{itemize}

\begin{lstlisting}
from roll import *
\end{lstlisting}

\end{frame}


\begin{frame}
\frametitle{Complete Coding Exercise}
Complete coding exercise is available at \url{https://github.com/vivek14632/Python-Workshop/tree/master/Introducing\%20Python/Chapter\%205/Code\%20Examples}
\end{frame}


\begin{frame}
\frametitle{Summary}
\begin{itemize}
\item We understood the use of modules in importing additional functionality into our code that gives us more firepower to work with.
\item Using command line arguments help us with more control over providing inputs to the code, while debugging especially.
\item Use of Packages to package all methods and classes together gives us more options to save the code for later reuse.
\end{itemize}
\end{frame}

\begin{frame}[fragile]
\frametitle{Exercise}

Comment all lines in \_\_init\_\_.py file and import the modules using following line of code.

\begin{lstlisting}
from roll import *
\end{lstlisting}

Now uncomment top two lines of \_\_init\_\_.py and  test the above import statement again. What difference did you notice?

Now, try the above exercise with following way of importing modules as discussed earlier.

\begin{lstlisting}
from roll import single, double
\end{lstlisting}

What difference did you notice in this case?

\end{frame}
\end{document}
